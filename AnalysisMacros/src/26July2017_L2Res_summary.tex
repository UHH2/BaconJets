\documentclass[t,compress]{beamer}
\mode<presentation>{
%\usetheme{CambridgeUS}
\usetheme{Madrid}
\usecolortheme{beaver}


}
\usepackage{graphicx} % Allows including images
\usepackage[percent]{overpic}
\usepackage{hyperref}
\usepackage{booktabs}
\usepackage{amsmath}
\usepackage{upgreek}
\usepackage{adjustbox}
\usepackage{setspace}

\usepackage{amsmath,amssymb}
\usepackage{color}

\usepackage{xcolor}
\usepackage{pict2e}


\title[L2 residual corrections]{L2 residual corrections derived on di-jet events}
\author[Jens Multhaup]{Anastasia Karavdina, Arne Reimers, \underline{Jens Multhaup}}
\institute[UHH]{University of Hamburg}

\date{\today}

\begin{document}

\begin{frame}
 \titlepage
\end{frame}


\begin{frame}
 \frametitle{Selection}
 \begin{columns}[t]
\begin{column}{.5\textwidth}
\begin{itemize}
 \item At least 2 jets with $p_T^{ave} > $ trigger thresholds
 \item Tag jet in barrel $|\eta| < 1.3$ taken as reference
 \item Probe jet in $0 < |\eta| < 5.2$ to be calibrated
 \item Leading jets are back to back ($\Delta \phi > 2.7$)
 \item Cut on relative third jet fraction $\alpha < 0.3$ \newline
    $\alpha =\frac{p_T^{jet3}}{p_T^{probe} + p_T^{barrel}}$
\end{itemize}
\end{column}
\begin{column}{.5\textwidth}
\begin{figure}
 \adjincludegraphics[width=1\linewidth,valign=t]{Bilder/DiJet.png}
\end{figure}
\end{column}
\end{columns}
\end{frame}



\begin{frame}
 \frametitle{Response calculation}
 \vspace{-1cm}
 \begin{columns}[t]
\begin{column}{.5\textwidth}
\begin{block}{$p_T$-balance response}
\begin{itemize}
\item For each event, store \newline
      \begin{equation}
            A = \frac{p_T^{probe} - p_T^{tag}}{p_T^{probe} + p_T^{tag}}
      \end{equation}
      in bins of $|\eta_{probe}|$ and $|p_T^{ave}|$
\item Calculate $<A>$ in bins of $|\eta_{probe}|$ and $|p_T^{ave}|$ for data and MC
\item Relative response: \newline
\begin{equation}
R_{rel}^{p_T} = \frac{1 + <A>}{1 - <A>}
\end{equation}
\end{itemize}
\end{block}
\end{column}
\begin{column}{.5\textwidth}
\begin{alertblock}{MPF response}
\begin{itemize}
\item For each event, store \newline
      \begin{equation}
            B = \frac{\vec{E_T} \cdot \vec{p_T^{tag}} /p_T^{tag}}{p_T^{probe} + p_T^{tag}} \newline
      \end{equation}
      in bins of $|\eta_{probe}|$ and $|p_T^{ave}|$
\item Calculate $<B>$ in bins of $|\eta_{probe}|$ and $|p_T^{ave}|$ for data and MC
\item Relative response:\newline
\begin{equation}
R_{rel}^{p_T} = \frac{1 + <B>}{1 - <B>} 
\end{equation}

\end{itemize}
\end{alertblock}
\end{column}
\end{columns}
\end{frame}

\begin{frame}
  \frametitle{Input}
  \setstretch{1.3}
  \vspace{-0.5cm}
  \begin{itemize}
   \item Data: /JetHT/Run2016*-03Feb2017/MINIAOD (reMINIAOD) 
   \item  MC: QCD Pt* TuneCUETP8M1 13TeV pythia8/RunIISummer16MiniAODv2-PUMoriond17 80X mcRun2 \newline
  \item JEC: Summer16 03Feb2017 \textcolor{red}{V3} Data
  \item JEC: Summer16 03Feb2017 V1 MC
  \item JER: JER Scaling factors and Uncertainty for 8 TeV (2012) \href{https://twiki.cern.ch/twiki/bin/viewauth/CMS/JetResolution}{[1]}
  \end{itemize}

More details:
\begin{itemize}
\item Lepton veto (Electron/Muon tight ID)
\item MET/$\Sigma p_T^{jets}$ cut not applied 
\item Eta-Phi cleaning (hotjets-runH.root)
\item New trigger setup 
\item changes in $p_T$ extrapolation
 \end{itemize}
 \end{frame}
 
 
\begin{frame}
  \frametitle{Trigger Setup}
  \vspace{-0.3cm}
\begin{itemize}
\item Added upper thresholds
\item Separate Central and FWD triggers at $\eta = 2.853$
\item except for: DiPFJetAve 40 and DiPFJetAve 60 \newline \hspace*{2cm} (also used in FWD region)
\item Dijet triggers and thresholds listed in \href{https://docs.google.com/spreadsheets/d/1LXIO3vw8RuMwBFgYUDtLmAghhFvWjs3gALRKlJxzWus/}{[2]}
 \end{itemize}
 \begin{columns}
  \begin{column}{0.5\textwidth}
  \vspace{-1.cm}
    \begin{figure}
 \adjincludegraphics[width=0.65\textwidth,valign=t]{Bilder/DIjetTriggers_ptAve_DATA.pdf}
\end{figure}
\vspace{-0.8cm}
    \begin{figure}
 \adjincludegraphics[width=0.65\textwidth,valign=t]{Bilder/CentralTrigger_pTave_Eta.pdf}
\end{figure}
  \end{column}
 \begin{column}{0.5\textwidth}
 \vspace{-1.cm}
   \begin{figure}
 \adjincludegraphics[width=0.65\textwidth,valign=t]{Bilder/DIjetTriggersHF_ptAve_DATA.pdf}
\end{figure}
\vspace{-0.8cm}
    \begin{figure}
 \adjincludegraphics[width=0.65\textwidth,valign=t]{Bilder/FWDTrigger_pTave_Eta.pdf}
\end{figure}
  \end{column}
 \end{columns}
 \end{frame}
 

\begin{frame}
\setstretch{1.3}
  \frametitle{Changes in $p_T$ extrapolation}
  \vspace{-0.5cm}
\begin{itemize}
\item extended $p_T$-range:  $[55; 1000]$
\item In all JEC versions: Use constant fit for $2.6 < |\eta| < 3.1$, \newline \hspace{2cm} $\to$ Expecting same (low) $p_T$ dependence as seen before 
\item Special treatment for $|\eta|>3.8$: Use same slope as in $3.4 < |\eta| < 3.8$, \newline \hspace{2cm} $\to$  Expecting same $p_T$ dependence as seen before
 \end{itemize}
  \begin{figure}
 \adjincludegraphics[width=0.6\textwidth,valign=t]{Bilder/img41.jpg}
\end{figure}
%  
% Reject high $p_T$ bins in some $\eta$ bins (noisy events)
% \begin{center}
% \begin{tabular}{|c|c|}
%  $\eta$ bin  & excluded $p_T$-bin \\ \hline \hline
%  2.8 - 2.9 & 365 - 453 \\
%            & 453 - 566 \\
%  2.9 - 3.1 & 453 - 566 \\
%  3.1 - 3.4 & 365 - 453 \\
%  3.4 - 3.8 & 230 - 299 \\
% \end{tabular}
% \end{center}
  \end{frame}
 

\begin{frame}
\frametitle{$p_T$ extrapolation with $p_T$-Balance}
\vspace{-0.5cm}
\begin{columns}[t]
\begin{column}{.33\textwidth}
\begin{figure}
 \begin{overpic}[width=0.9\linewidth]{Bilder/H/pTextrapolation_Pt_pythia8_pT_2650_2853.pdf}
  \put(25,20){constant fit}
 \end{overpic}
\end{figure}
% \adjincludegraphics[width=0.9\linewidth,valign=t]{Bilder/H/pTextrapolation_Pt_pythia8_pT_2650_2853.pdf}
\adjincludegraphics[width=0.9\linewidth,valign=t]{Bilder/H/pTextrapolation_Pt_pythia8_pT_3139_3489.pdf}
 \end{column}
 \begin{column}{.33\textwidth}
 \begin{figure}
 \begin{overpic}[width=0.9\linewidth]{Bilder/H/pTextrapolation_Pt_pythia8_pT_2853_2964.pdf}
  \put(25,20){constant fit}
 \end{overpic}
\end{figure}
% \adjincludegraphics[width=0.9\linewidth,valign=t]{Bilder/H/pTextrapolation_Pt_pythia8_pT_2853_2964.pdf}
\adjincludegraphics[width=0.9\linewidth,valign=t]{Bilder/H/pTextrapolation_Pt_pythia8_pT_3489_3839.pdf}
 \end{column} 
\begin{column}{.33\textwidth}
\begin{figure}
 \begin{overpic}[width=0.9\linewidth]{Bilder/H/pTextrapolation_Pt_pythia8_pT_2964_3139.pdf}
  \put(25,20){constant fit}
 \end{overpic}
\end{figure}
\vspace{-0.25cm}
% \adjincludegraphics[width=0.9\linewidth,valign=t]{Bilder/H/pTextrapolation_Pt_pythia8_pT_2964_3139.pdf}
\begin{figure}
 \begin{overpic}[width=0.9\linewidth]{Bilder/H/pTextrapolation_Pt_pythia8_pT_3839_5191.pdf}
  \put(30,18){same slope}
 \end{overpic}
\end{figure}
% \adjincludegraphics[width=0.9\linewidth,valign=t]{Bilder/H/pTextrapolation_Pt_pythia8_pT_3839_5191.pdf}
\end{column}
\end{columns}
\end{frame}

 
\begin{frame}
\frametitle{All Runs: L2Res Corrections before kFSR}
\vspace{-1cm}
\begin{columns}
 \begin{column}{0.5\textwidth}
 \begin{figure}
 \adjincludegraphics[width=0.7\textwidth,valign=t]{Bilder/BCD/Ratio_AK4PFchs_pythia8.pdf}
\end{figure}
\vspace{-1cm}
\begin{figure}
 \adjincludegraphics[width=0.7\textwidth,valign=t]{Bilder/FlateG/Ratio_AK4PFchs_pythia8.pdf}
\end{figure}
\end{column}
\begin{column}{0.5\textwidth}
\begin{figure}
 \adjincludegraphics[width=0.7\textwidth]{Bilder/EFearly/Ratio_AK4PFchs_pythia8.pdf}
\end{figure}
\vspace{-1cm}
\begin{figure}
 \adjincludegraphics[width=0.7\textwidth]{Bilder/H/Ratio_AK4PFchs_pythia8.pdf}
\end{figure}
 \end{column}
 \end{columns}
 \end{frame}
 
\begin{frame}
\frametitle{All Runs: kFSR extrapolation}
\vspace{-1cm}
\begin{columns}
 \begin{column}{0.5\textwidth}
 \begin{figure}
 \adjincludegraphics[width=0.7\textwidth,valign=t]{Bilder/BCD/kFSR_AK4PFchs_pythia8.pdf}
\end{figure}
\vspace{-1cm}
\begin{figure}
 \adjincludegraphics[width=0.7\textwidth,valign=t]{Bilder/FlateG/kFSR_AK4PFchs_pythia8.pdf}
\end{figure}
\end{column}
\begin{column}{0.5\textwidth}
\begin{figure}
 \adjincludegraphics[width=0.7\textwidth]{Bilder/EFearly/kFSR_AK4PFchs_pythia8.pdf}
\end{figure}
\vspace{-1cm}
\begin{figure}
 \adjincludegraphics[width=0.7\textwidth]{Bilder/H/kFSR_AK4PFchs_pythia8.pdf}
\end{figure}
 \end{column}
 \end{columns}
 \end{frame}
 
%  
%  \begin{frame}
% \frametitle{All Runs: kFSR extrapolation}
% \begin{itemize}
%  \item MPF method: fit-function looks good 
%  \item Discrepancy for $p_T$-balance in HF ($|\eta|>2.6$)
%  \item Use fit values for L2Res calculation, \newline \hspace{1cm} $\to$ function from RunI, \newline \hspace{1cm} $\to$ Works quite well, also for $p_T$-balance method, 
%  \item Still under investigation (MC generator, underlying distributions)
% \end{itemize}
% \begin{columns}
%  \begin{column}{0.4\textwidth}
%  \vspace*{1.5cm}\newline
% Usage of histogram values instead of fit values \newline \hspace{1cm} $\to$ Bias in L2Res?
%  \end{column}
% \begin{column}{0.6\textwidth}
% \vspace{-0.8cm}
%  \begin{figure}
%  \begin{overpic}[width=0.7\textwidth]{Bilder/H/L2Res_kFSRval_AK4PFchs_pythia8.pdf}
%   \put(18,72){\tiny{kFSR hist values used}}
%  \end{overpic}
% \end{figure}
% \end{column}
% \end{columns}
% \end{frame}
%  
 
\begin{frame}
\frametitle{All Runs: L2Res corrections with kFSR}
\vspace{-1cm}
\begin{columns}
 \begin{column}{0.5\textwidth}
 \begin{figure}
 \adjincludegraphics[width=0.7\textwidth,valign=t]{Bilder/BCD/L2Res_kFSRval_AK4PFchs_pythia8.pdf}
\end{figure}
\vspace{-1cm}
\begin{figure}
 \adjincludegraphics[width=0.7\textwidth,valign=t]{Bilder/FlateG/L2Res_kFSRval_AK4PFchs_pythia8.pdf}
\end{figure}
\end{column}
\begin{column}{0.5\textwidth}
\begin{figure}
 \adjincludegraphics[width=0.7\textwidth]{Bilder/EFearly/L2Res_kFSRval_AK4PFchs_pythia8.pdf}
\end{figure}
\vspace{-1cm}
\begin{figure}
 \adjincludegraphics[width=0.7\textwidth]{Bilder/H/L2Res_kFSRval_AK4PFchs_pythia8.pdf}
\end{figure}
 \end{column}
 \end{columns}
 \end{frame}
%  
% \begin{frame}
% \frametitle{$p_T$ dependence for MPF: log-lin fit}
% \vspace{-1cm}
% \begin{columns}
%  \begin{column}{0.5\textwidth}
%  \begin{figure}
%  \adjincludegraphics[width=0.7\textwidth,valign=t]{Bilder/BCD/L2Res_logpt_MPF_kFSRfit_AK4PFchs_pythia8.pdf}
% \end{figure}
% \vspace{-1cm}
% \begin{figure}
%  \adjincludegraphics[width=0.7\textwidth,valign=t]{Bilder/FlateG/L2Res_logpt_MPF_kFSRfit_AK4PFchs_pythia8.pdf}
% \end{figure}
% \end{column}
% \begin{column}{0.5\textwidth}
% \begin{figure}
%  \adjincludegraphics[width=0.7\textwidth]{Bilder/EFearly/L2Res_logpt_MPF_kFSRfit_AK4PFchs_pythia8.pdf}
% \end{figure}
% \vspace{-1cm}
% \begin{figure}
%  \adjincludegraphics[width=0.7\textwidth]{Bilder/H/L2Res_logpt_MPF_kFSRfit_AK4PFchs_pythia8.pdf}
% \end{figure}
%  \end{column}
%  \end{columns}
%  \end{frame}
%  
% \begin{frame}
% \frametitle{$p_T$ dependence for $p_T$-balance: log-lin fit}
% \vspace{-1cm}
% \begin{columns}
%  \begin{column}{0.5\textwidth}
%  \begin{figure}
%  \adjincludegraphics[width=0.7\textwidth,valign=t]{Bilder/BCD/L2Res_logpt_DiJet_kFSRfit_AK4PFchs_pythia8.pdf}
% \end{figure}
% \vspace{-1cm}
% \begin{figure}
%  \adjincludegraphics[width=0.7\textwidth,valign=t]{Bilder/FlateG/L2Res_logpt_DiJet_kFSRfit_AK4PFchs_pythia8.pdf}
% \end{figure}
% \end{column}
% \begin{column}{0.5\textwidth}
% \begin{figure}
%  \adjincludegraphics[width=0.7\textwidth]{Bilder/EFearly/L2Res_logpt_DiJet_kFSRfit_AK4PFchs_pythia8.pdf}
% \end{figure}
% \vspace{-1cm}
% \begin{figure}
%  \adjincludegraphics[width=0.7\textwidth]{Bilder/H/L2Res_logpt_DiJet_kFSRfit_AK4PFchs_pythia8.pdf}
% \end{figure}
%  \end{column}
%  \end{columns}
%  \end{frame}
%  
%  
%  \begin{frame}
% \frametitle{$p_T$ dependence: log-lin fit}
% \begin{itemize}
%  \item Small $p_T$ dependence in barrel part
%  \item high dependence for $|\eta|>2.8$ 
%  \item Provide corrections up to last filled $p_T$ bin \newline $\to$ beyond: Corrections stay const. 
% \end{itemize}
% \begin{columns}
%  \begin{column}{0.5\textwidth}
%  \begin{figure}
%  \begin{overpic}[width=0.8\textwidth]{Bilder/H/L2Res_logpt_MPF_kFSRfit_AK4PFchs_pythia8.pdf}
%   \put(20,70){\tiny{MPF}}
%  \end{overpic}
% \end{figure}
% \end{column}
% \begin{column}{0.5\textwidth}
%  \begin{figure}
%  \begin{overpic}[width=0.8\textwidth]{Bilder/H/L2Res_logpt_DiJet_kFSRfit_AK4PFchs_pythia8.pdf}
%   \put(20,70){\tiny{$p_T$ balance}}
%  \end{overpic}
% \end{figure}
%  \end{column}
%  \end{columns}
% 
%  \end{frame}
%  
 
 \begin{frame}
\frametitle{Summary: L2Res calculation}
\vspace{-0.5cm}
\begin{itemize}
 \item Eta-Phi cleaning
 \item New tirgger setup
 \item Changes in $p_T$ extrapolation
 \item kFSR discrepancy for $p_T$-balance method
 \item Calculated L2Res corrections vs. $|\eta|$
 \item Provided new JEC version (V7)
 \end{itemize}
 Difference:
\begin{itemize}
 \item Version 7: $p_T$ -balance method (full $\eta$ range) kFSR hist values used 
 \end{itemize}
 \end{frame} 
%   
%   \begin{frame}
% \frametitle{Closure Test}
% L2Res corrections calculated for $|\eta|$ \newline
% \vspace{0.5cm}
% Perform Closure Test for $\eta$:
% \begin{itemize}
%  \item Apply one JEC version (V4, V5 and V6)
%  \item Try to calculate L2Res corrections again (for $\eta$ and $|\eta|$)
%  \item Expect them to be $\approx 1$
% \end{itemize}
% \vspace{1cm}
% Difference between the 3 JEC versions:
% \begin{itemize}
%  \item Version 4: MPF method (full $\eta$ range)
%  \item Version 5: $p_T$ -balance method (full $\eta$ range)
%  \item Version 6: MPF method ($|\eta|<2.5$)\newline \hspace*{1.7cm}$p_T$-balance method ($|\eta|>2.5$) 
% \end{itemize}
% \end{frame} 
%  
% \begin{frame}
% \frametitle{Run H: Closure Test for V4}
% \vspace{-0.3cm}
% $p_T$ dependence: log-lin fit
% \vspace{-0.8cm}
% \begin{columns}
%  \begin{column}{0.5\textwidth}
%  \begin{figure}
%  \begin{overpic}[width=0.7\textwidth]{Bilder/H_Closure_V4/L2Res_logpt_MPF_kFSRfit_AK4PFchs_pythia8.pdf}
%   \put(40,70){\tiny{MPF}}
%  \end{overpic}
% \end{figure}
% \vspace{-1cm}
%  \begin{figure}
%  \begin{overpic}[width=0.7\textwidth]{Bilder/H_Closure_V4/abs_eta/L2Res_logpt_MPF_kFSRfit_AK4PFchs_pythia8.pdf}
%   \put(40,70){\tiny{MPF}}
%  \end{overpic}
% \end{figure}
% \end{column}
% \begin{column}{0.5\textwidth}
%  \begin{figure}
%  \begin{overpic}[width=0.7\textwidth]{Bilder/H_Closure_V4/L2Res_logpt_DiJet_kFSRfit_AK4PFchs_pythia8.pdf}
%   \put(40,70){\tiny{$p_T$ balance}}
%  \end{overpic}
% \end{figure}
% \vspace{-1cm}
%  \begin{figure}
%  \begin{overpic}[width=0.7\textwidth]{Bilder/H_Closure_V4/abs_eta/L2Res_logpt_DiJet_kFSRfit_AK4PFchs_pythia8.pdf}
%   \put(20,70){\tiny{$p_T$ balance}}
%  \end{overpic}
% \end{figure}
%  \end{column}
%  \end{columns}
%  \end{frame} 
% 
%  
% \begin{frame}
% \frametitle{Run H: Closure Test for V5}
% \vspace{-0.3cm}
% $p_T$ dependence: log-lin fit
% \vspace{-0.8cm}
% \begin{columns}
%  \begin{column}{0.5\textwidth}
%  \begin{figure}
%  \begin{overpic}[width=0.7\textwidth]{Bilder/H_Closure_V5/L2Res_logpt_MPF_kFSRfit_AK4PFchs_pythia8.pdf}
%   \put(40,70){\tiny{MPF}}
%  \end{overpic}
% \end{figure}
% \vspace{-1cm}
%  \begin{figure}
%  \begin{overpic}[width=0.7\textwidth]{Bilder/H_Closure_V5/abs_eta/L2Res_logpt_MPF_kFSRfit_AK4PFchs_pythia8.pdf}
%   \put(40,70){\tiny{MPF}}
%  \end{overpic}
% \end{figure}
% \end{column}
% \begin{column}{0.5\textwidth}
%  \begin{figure}
%  \begin{overpic}[width=0.7\textwidth]{Bilder/H_Closure_V5/L2Res_logpt_DiJet_kFSRfit_AK4PFchs_pythia8.pdf}
%   \put(40,70){\tiny{$p_T$ balance}}
%  \end{overpic}
% \end{figure}
% \vspace{-1cm}
%  \begin{figure}
%  \begin{overpic}[width=0.7\textwidth]{Bilder/H_Closure_V5/abs_eta/L2Res_logpt_DiJet_kFSRfit_AK4PFchs_pythia8.pdf}
%   \put(20,70){\tiny{$p_T$ balance}}
%  \end{overpic}
% \end{figure}
%  \end{column}
%  \end{columns}
%  \end{frame} 
%     
% \begin{frame}
% \frametitle{Summary}
% \begin{columns}
%  \begin{column}{0.5\textwidth}
%  Closure for $|\eta|$ \newline
% \begin{itemize}
% \item Good closure over full range
% \item In $2.8 < |\eta| < 3.1$: High $p_T$ dependence (constant fit)
% \end{itemize}
%  \end{column}
% \begin{column}{0.5\textwidth}
%   Closure for $\eta$ \newline
% \begin{itemize}
% \item Good closure except $-3.1 < \eta < 2.8$ and $2.8 < \eta < 3.1$
% \item Observed L2Res difference in this region before ($|\eta| \to \eta$)
% \item In $2.6 < |\eta| < 3.1$: High $p_T$ dependence (constant fit)
% \end{itemize}
%  \end{column}
% \end{columns}
% \begin{center}
% \end{center}
% \end{frame}


\title{Backup}
\author{}
\institute{}
\date{}

\begin{frame}
\maketitle
\end{frame}
% 
%  
% \begin{frame}
% \frametitle{Run BCD: Closure Test for V4}
% \vspace{-1cm}
% \begin{columns}
%  \begin{column}{0.5\textwidth}
%  \begin{figure}
%  \adjincludegraphics[width=0.7\textwidth,valign=t]{Bilder/BCD_Closure_V4/L2Res_logpt_MPF_kFSRfit_AK4PFchs_pythia8.pdf}
% \end{figure}
% \vspace{-1cm}
% \begin{figure}
%  \adjincludegraphics[width=0.7\textwidth,valign=t]{Bilder/BCD_Closure_V4/abs_eta/L2Res_logpt_MPF_kFSRfit_AK4PFchs_pythia8.pdf}
% \end{figure}
% \end{column}
% \begin{column}{0.5\textwidth}
% \begin{figure}
%  \adjincludegraphics[width=0.7\textwidth]{Bilder/BCD_Closure_V4/L2Res_logpt_DiJet_kFSRfit_AK4PFchs_pythia8.pdf}
% \end{figure}
% \vspace{-1cm}
% \begin{figure}
%  \adjincludegraphics[width=0.7\textwidth]{Bilder/BCD_Closure_V4/abs_eta/L2Res_logpt_DiJet_kFSRfit_AK4PFchs_pythia8.pdf}
% \end{figure}
%  \end{column}
%  \end{columns}
%  \end{frame} 
% 
%  \begin{frame}
% \frametitle{Run BCD: Closure Test for V5}
% \vspace{-1cm}
% \begin{columns}
%  \begin{column}{0.5\textwidth}
%  \begin{figure}
%  \adjincludegraphics[width=0.7\textwidth,valign=t]{Bilder/BCD_Closure_V5/L2Res_logpt_MPF_kFSRfit_AK4PFchs_pythia8.pdf}
% \end{figure}
% \vspace{-1cm}
% \begin{figure}
%  \adjincludegraphics[width=0.7\textwidth,valign=t]{Bilder/BCD_Closure_V5/abs_eta/L2Res_logpt_MPF_kFSRfit_AK4PFchs_pythia8.pdf}
% \end{figure}
% \end{column}
% \begin{column}{0.5\textwidth}
% \begin{figure}
%  \adjincludegraphics[width=0.7\textwidth]{Bilder/BCD_Closure_V5/L2Res_logpt_DiJet_kFSRfit_AK4PFchs_pythia8.pdf}
% \end{figure}
% \vspace{-1cm}
% \begin{figure}
%  \adjincludegraphics[width=0.7\textwidth]{Bilder/BCD_Closure_V5/abs_eta/L2Res_logpt_DiJet_kFSRfit_AK4PFchs_pythia8.pdf}
% \end{figure}
%  \end{column}
%  \end{columns}
%  \end{frame} 
% 
% \begin{frame}
% \frametitle{Run EFearly: Closure Test for V4}
% \vspace{-1cm}
% \begin{columns}
%  \begin{column}{0.5\textwidth}
%  \begin{figure}
%  \adjincludegraphics[width=0.7\textwidth,valign=t]{Bilder/EFearly_Closure_V4/L2Res_logpt_MPF_kFSRfit_AK4PFchs_pythia8.pdf}
% \end{figure}
% \vspace{-1cm}
% \begin{figure}
%  \adjincludegraphics[width=0.7\textwidth,valign=t]{Bilder/EFearly_Closure_V4/abs_eta/L2Res_logpt_MPF_kFSRfit_AK4PFchs_pythia8.pdf}
% \end{figure}
% \end{column}
% \begin{column}{0.5\textwidth}
% \begin{figure}
%  \adjincludegraphics[width=0.7\textwidth]{Bilder/EFearly_Closure_V4/L2Res_logpt_DiJet_kFSRfit_AK4PFchs_pythia8.pdf}
% \end{figure}
% \vspace{-1cm}
% \begin{figure}
%  \adjincludegraphics[width=0.7\textwidth]{Bilder/EFearly_Closure_V4/abs_eta/L2Res_logpt_DiJet_kFSRfit_AK4PFchs_pythia8.pdf}
% \end{figure}
%  \end{column}
%  \end{columns}
%  \end{frame} 
% 
%  \begin{frame}
% \frametitle{Run EFearly: Closure Test for V5}
% \vspace{-1cm}
% \begin{columns}
%  \begin{column}{0.5\textwidth}
%  \begin{figure}
%  \adjincludegraphics[width=0.7\textwidth,valign=t]{Bilder/EFearly_Closure_V5/L2Res_logpt_MPF_kFSRfit_AK4PFchs_pythia8.pdf}
% \end{figure}
% \vspace{-1cm}
% \begin{figure}
%  \adjincludegraphics[width=0.7\textwidth,valign=t]{Bilder/EFearly_Closure_V5/abs_eta/L2Res_logpt_MPF_kFSRfit_AK4PFchs_pythia8.pdf}
% \end{figure}
% \end{column}
% \begin{column}{0.5\textwidth}
% \begin{figure}
%  \adjincludegraphics[width=0.7\textwidth]{Bilder/EFearly_Closure_V5/L2Res_logpt_DiJet_kFSRfit_AK4PFchs_pythia8.pdf}
% \end{figure}
% \vspace{-1cm}
% \begin{figure}
%  \adjincludegraphics[width=0.7\textwidth]{Bilder/EFearly_Closure_V5/abs_eta/L2Res_logpt_DiJet_kFSRfit_AK4PFchs_pythia8.pdf}
% \end{figure}
%  \end{column}
%  \end{columns}
%  \end{frame} 
%  
%  
% \begin{frame}
% \frametitle{Run FlateG: Closure Test for V4}
% \vspace{-1cm}
% \begin{columns}
%  \begin{column}{0.5\textwidth}
%  \begin{figure}
%  \adjincludegraphics[width=0.7\textwidth,valign=t]{Bilder/FlateG_Closure_V4/L2Res_logpt_MPF_kFSRfit_AK4PFchs_pythia8.pdf}
% \end{figure}
% \vspace{-1cm}
% \begin{figure}
%  \adjincludegraphics[width=0.7\textwidth,valign=t]{Bilder/FlateG_Closure_V4/abs_eta/L2Res_logpt_MPF_kFSRfit_AK4PFchs_pythia8.pdf}
% \end{figure}
% \end{column}
% \begin{column}{0.5\textwidth}
% \begin{figure}
%  \adjincludegraphics[width=0.7\textwidth]{Bilder/FlateG_Closure_V4/L2Res_logpt_DiJet_kFSRfit_AK4PFchs_pythia8.pdf}
% \end{figure}
% \vspace{-1cm}
% \begin{figure}
%  \adjincludegraphics[width=0.7\textwidth]{Bilder/FlateG_Closure_V4/abs_eta/L2Res_logpt_DiJet_kFSRfit_AK4PFchs_pythia8.pdf}
% \end{figure}
%  \end{column}
%  \end{columns}
%  \end{frame} 
% 
%  \begin{frame}
% \frametitle{Run FlateG: Closure Test for V5}
% \vspace{-1cm}
% \begin{columns}
%  \begin{column}{0.5\textwidth}
%  \begin{figure}
%  \adjincludegraphics[width=0.7\textwidth,valign=t]{Bilder/FlateG_Closure_V5/L2Res_logpt_MPF_kFSRfit_AK4PFchs_pythia8.pdf}
% \end{figure}
% \vspace{-1cm}
% \begin{figure}
%  \adjincludegraphics[width=0.7\textwidth,valign=t]{Bilder/FlateG_Closure_V5/abs_eta/L2Res_logpt_MPF_kFSRfit_AK4PFchs_pythia8.pdf}
% \end{figure}
% \end{column}
% \begin{column}{0.5\textwidth}
% \begin{figure}
%  \adjincludegraphics[width=0.7\textwidth]{Bilder/FlateG_Closure_V5/L2Res_logpt_DiJet_kFSRfit_AK4PFchs_pythia8.pdf}
% \end{figure}
% \vspace{-1cm}
% \begin{figure}
%  \adjincludegraphics[width=0.7\textwidth]{Bilder/FlateG_Closure_V5/abs_eta/L2Res_logpt_DiJet_kFSRfit_AK4PFchs_pythia8.pdf}
% \end{figure}
%  \end{column}
%  \end{columns}
%  \end{frame} 
%  


\end{document}
